
\chapter{Diskrete Standortplanung} % (fold)
\label{cha:diskrete_standortplanung}

  \section{Klassifikation diskreter Standortprobleme} % (fold)
  \label{sec:klassifikation_diskreter_standortprobleme}

    \par \textbf{Planungshorizont}
    \par Anzahl der Zeitperioden (z. B. Monate, Jahre) im Planungszeitraum.
    \begin{itemize}
      \item Statische (Ein-Perioden) Modelle\\
      Standortentscheidungen werden zu Beginn des Planungshorizontes getroffen
      \item Dynamische (Mehr-Perioden) Modelle\\
      Entscheide wo und wann, d. h. in welcher Periode, Einrichtungen platziert werden.
    \end{itemize}

    \par \textbf{Einrichtungstypen}
    \par Unterscheide verschiedene Ausprägungen von Einrichtungen,

    \par \textbf{Produktaggregation}
    \par Ist nur ein Produkt oder sind mehrere Produkte mit unterschiedlichen Charakteristiken zu planen?

    \par \textbf{Stufigkeit}
    \par Anzahl der Stufen des Distributionsnetzwerks.
    \begin{itemize}
      \item Einstufige Modelle\\
      Güterverteilung über eine Transportstufe (z. B. Lager$\rightarrow$Händler). Standortentscheidung nur auf einer Ebene (z. B. Lager).
      \item Mehrstufige (hierarchische) Modelle\\
      Güterverteilung über mehrere Transportstufen (z. B. Werk$\rightarrow$Lager$\rightarrow$Händler).
    \end{itemize}

    \par \textbf{Interaktion}
    \par Interaktion zwischen Einrichtungen desselben Typs möglich, z. B. Transporte zwischen Warenhäusern.

    \par \textbf{Unsicherheit}
    \begin{itemize}
      \item Deterministische Modelle\\
      alle Daten und Parameter sind vollständig bekannt
      \item Probabilistische (stochastische) Modelle\\
      manche Daten und Parameter sind unsicher
    \end{itemize}

    \par \textbf{Flussrichtung}
    \begin{itemize}
      \item Distributionslogistik\\
      Waren fließen von Werken/Lieferanten zu den Lagern/Kunden
      \item Entsorgungslogistik (Reverse Logistics)\\
      Waren fließen von Kunden zu Recyclinganlagen, Deponien
    \end{itemize}

    \par \textbf{Kapazitätsrestriktionen}
    \par Beschränken die Mengen, welche produziert, transportiert, umgeschlagen etc. werden können.

    \par \textbf{Transportmodus}
    \begin{itemize}
      \item Touren-Belieferungen\\ 
      Auslieferungen von einer übergeordneten Einrichtung zu untergeordneten Einrichtungen (oder umgekehrt) werden zu Touren gebündelt.
      \item Direkt-Belieferungen\\ 
      Auslieferungen von übergeordneten zu untergeordneten Ebenen geschehen direkt und ohne Umwege.
    \end{itemize}
    
    \par \textbf{}


  
  % section klassifikation_diskreter_standortprobleme (end)

  \section{Das Warehouse Location Problem (WLP)} % (fold)
  \label{sec:das_warehouse_location_problem_}

    \par Auch als \textbf{Uncapacitated Facility Location Problem (UFLP)} oder \textbf{Simple Plant Location Problem (SPLP)} bekannt.

    \par \textbf{Modell-Annahmen}
    \begin{itemize}
      \item statisch, einstufig, unkapazitiert, deterministisch
      \item ein Produkt, Direkt-Belieferung, ein Einrichtungstyp, keine Interaktion
    \end{itemize}

    \par \textbf{Gegeben}
    \begin{itemize}
      \item Menge von Kunden $I = \{1, \dots, n\}$ mit Bedarfen $b_i, i = 1, \dots, n$
      \item Menge potentieller Standorte $J = \{1, \dots, m\}$ für neue Einrichtungen
      \item Kosten $t_{ij}$ für den Transport einer Mengeneinheit vom potentiellen Standort $j \in J$ zu einem Kunden $i \in I$.
      $\rightarrow$ Gesamtkosten zur Belieferung des Kunden $i$ vom Standort $j$: $c_{ij} = b_i \cdot t_{ij}$ \\$\rightarrow$ Kostenmatrix $C = (c_{ij})_{i = 1, \dots, n, j = 1, \dots, m}$
      \item Fixkosten $f_j$ für die Errichtung einer neuen Einrichtung am Standort $j$.
    \end{itemize}

    \par \textbf{Entscheidungen}
    \begin{itemize}
      \item Anzahl neuer Einrichtungen
      \item Standorte neuer Einrichtungen
      \item Zuordnung von Kunden zu neuen Einrichtungen
    \end{itemize}


    \subsection{Modellierug} % (fold)
    \label{sub:modellierug}

      \par \textbf{Standortentscheidung}
      $$
        y_j = 
        \begin{cases}
          1 & \text{falls am potentiellen Standort } j \text{ eine neue Einrichtung platziert wird}, j \in J\\
          0 & \text{sonst}
        \end{cases} 
      $$

      \par \textbf{Zuordnungsentscheidung}
    
      $$
        x_{ij} = 
        \begin{cases}
          1 & \text{falls Kunde i einer neuen Einrichtung am Standort } j \text{ zugeordnet wird}, i \in I, j \in J\\
          0 & \text{sonst}
        \end{cases} 
      $$

      \par \textbf{Mathematisches Modell}

      \begin{equation}
        \begin{aligned}
          & \underset{}{\text{min}}
          && \sum_{i=1}^{n}\sum_{j=1}^{m}c_{ij}x_{ij} + \sum_{j=1}^{m}f_ky_{j}\\
          % & & e - (s + re) \\
          & \text{u.d.N}
          & & \sum_{j=1}^{m}x_{ij}=1, \forall i \qquad  (\text{Jeder Kunde } i \text{ wird genau einer neuen Einrichtung an einem Standort } j \text{ zugeordnet})\\ 
          & & & x_{ij} \leq y_j, \forall i, \forall j \quad \text{(Kunde } i \text{ darf nur dann Standort } j \text{ zugeordnet werden, wenn an } j \text{ eine Einrichtung platziert)}\\ 
          & & & x_{ij} \geq 0, y_j \in \{0,1\}, \forall i, \forall j \\
        \end{aligned}
      \end{equation}

    % subsection modellierug (end)

    \subsection{Heuristiken} % (fold)
    \label{sub:heuristiken}

      Methoden zur Bestimmung einer heuristischen Lösung $x_H$ für ein Problem.

      (Finden nicht notwendigerweise eine optimale Lösung.)

      \par \textbf{}
      

      \subsubsection{Greedy-Heuristik}

      \subsubsection{Interchange-Heuristik}

    % subsection heuristiken (end)

    \subsection{Das DUALOC-Verfahren} % (fold)
    \label{sub:das_dualoc_verfahren}
    
    % subsection das_dualoc_verfahren (end)

  % section das_warehouse_location_problem_ (end)

  \section{Hub-Location-Probleme} % (fold)
  \label{sec:hub_location_probleme}
  
  % section hub_location_probleme (end)
% chapter diskrete_standortplanung (end)

    


